%% Author_tex.tex
%% V1.1
%% 2012/18/6
%% Revised on 2015/20/1
%%
%% developed by Techset
%%
%% This file describes the coding for ptephy_v1.cls

%\documentclass{ptephy_v1}%%%%where ptephy_v1 is the template name
\documentclass[preprint]{ptephy_v1}%%%%%% to generate preprint number
%\documentclass{ptephy_v1}%%%%%% to generate preprint number with ptep logo

\preprintnumber{XXXX-XXXX} %%% %%% Insert preprint number here

%The authors can define any packages after the \documentclass{ptephy_v1} command.

\usepackage[T1]{fontenc}
\usepackage[utf8]{inputenc}

\usepackage{ifxetex}
\ifxetex
\usepackage{xeCJK}
\xeCJKsetup{CJKmath=true}
\newcommand{\jp}[1]{#1}
\else
%\usepackage[T1]{CJKutf8}
\usepackage[whole]{bxcjkjatype}
\newcommand{\jp}[1]{\text{#1}}
\fi

\usepackage{amsmath}
\usepackage{amsthm}
\usepackage{hyperref}
\usepackage{graphics}
%\usepackage{algorithmic} for describing algorithms
%\usepackage{subfig} for getting the subfigures e.g., "Figure 1a and 1b" etc.
\usepackage{url}
\usepackage{tikz}
\usetikzlibrary{positioning,arrows.meta}
%\usepackage{sagetex}

%The author can find the documentation of additional supporting files from "http://www.ctan.org"

% *** Do not adjust lengths that control margins, column widths, etc. ***

\newtheorem{axiom}{Axiom}
\newtheorem{dfn}{Definition}
\newtheorem{theorem}{Theorem}
\newtheorem{condition}{Condition}

\begin{document}
%\begin{CJK}{UTF8}{ipxm}

\title{Constructing Particle out from Space}

%%%% To generate auto affiliation numbers please use \author{}\affil{} command

\author{Doumu/Fudepia}
\affil{\email{fudepia@outlook.jp}}

\begin{abstract}
    In this paper, we'll try to construct particles out from the fabric of the universe.
\end{abstract}


\subjectindex{xxxx, xxx}

\maketitle



\section{The Axiom of Observation}
\begin{axiom}[Observation]
    We could only observe things that are different from their surroundings.
\end{axiom}

Basically what this means is, if given a thing with all its' properties uniformly distributed, they are no way you could observe things inside that thing. And if one argues that ``But sure you could observe that thing as a whole.'', then the explanation is that: that thing as a whole, is different from the background you're observing from. Actually, from here we could also define what's so called boundary.

\begin{dfn}[Boundary]
    A line with certain thickness (most time, infinitesimal), such that at two ends of the thickness, the difference between them are significant enough.
\end{dfn}

Within computer science, normally we meant that the difference is greater that so called EPS (epsilon). But the idea could further extend to physics as something that's limited by uncertainty principle.

\section{Existence and Roots}
Not going into detail on how the whole idea of existence is derived, existence $\jp{エ}(x)$ basically represents the unit-length of that certain point $x$. So to measure an object's certain property at $x$, given the property is observed by the object itself as $k$, what you'll observe is $k\frac{\jp{エ}(yourself)}{\jp{エ}(object)}$ where normally $\jp{エ}(yourself)=1$ (which basically meant the unit-length of the object $,\jp{エ}(object)$ is measured by your unit-length).~\cite{mucoe}

Now here is an obvious issue (or what we considered as feature), that if you measured the unit-length of a certain object regarding certain property to be $0$, then you got a division by zero.

\begin{theorem}[Roots of \jp{エ}(x)]
    <+content+>
    \label{<+label+>}
\end{theorem}<++>

\section{Difference and Existence}
\begin{theorem}[Approximation on Existence]
    A unit-length should ensure the space's changes that it's describing is meaningful.
    \label{thm:approxExist}
\end{theorem}

Yeah, a very poetic and vague description. So put it in a more mathematical and rigid, \textit{\textbf{being meaningful}} just means a range that isn't too ginormous, yet also not too minuscule. Combining it to that we're describing a spaces changes (at that local region the existence is describing), it meant if properties changes dramatically over that region, it'll need a large existence, and vice versa.


% \section{Conclusion}

% \section*{Acknowledgment}

% can use a bibliography generated by BibTeX as a .bbl file
% BibTeX documentation can be easily obtained at:
% http://www.ctan.org/tex-archive/biblio/bibtex/contrib/doc/

%\bibliographystyle{ptephy}
%\bibliography{sample}
%
% once the .bbl file has been generated then place the text in your article.

%This is added by T. Yoneya (editor-in-chief) on 2020/07/09.

\let\doi\relax

%without this code before the command "\begin{thebibliography}{}" , an error will be %flagged. When the bibliography is provided as separate .bib file, then this code %should be placed above the commands "\bibliographystyle{}" and "\bibliography{}" %inside the main TeX file. 

\begin{thebibliography}{9}

    \bibitem{mucoe}
        \url{https://www.mail-archive.com/dou-geometry@googlegroups.com/msg00004/____________-___________________________.docx}

\end{thebibliography}

\appendix





%\end{CJK}
\end{document}
