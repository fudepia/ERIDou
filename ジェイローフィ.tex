%% Author_tex.tex
%% V1.1
%% 2012/18/6
%% Revised on 2015/20/1
%%
%% developed by Techset
%%
%% This file describes the coding for ptephy_v1.cls

%\documentclass{ptephy_v1}%%%%where ptephy_v1 is the template name
\documentclass[preprint]{ptephy_v1}%%%%%% to generate preprint number
%\documentclass{ptephy_v1}%%%%%% to generate preprint number with ptep logo

\preprintnumber{XXXX-XXXX} %%% %%% Insert preprint number here

%The authors can define any packages after the \documentclass{ptephy_v1} command.

\usepackage[T1]{fontenc}
\usepackage[utf8]{inputenc}

\usepackage{ifxetex}
\ifxetex
\usepackage{xeCJK}
\xeCJKsetup{CJKmath=true}
\newcommand{\jp}[1]{#1}
\else
%\usepackage[T1]{CJKutf8}
\usepackage[whole]{bxcjkjatype}
\newcommand{\jp}[1]{\text{#1}}
\fi

\usepackage{amsmath}
\usepackage{amsthm}
\usepackage{hyperref}
\usepackage{graphics}
%\usepackage{algorithmic} for describing algorithms
%\usepackage{subfig} for getting the subfigures e.g., "Figure 1a and 1b" etc.
\usepackage{url}
\usepackage{tikz}
\usetikzlibrary{positioning,arrows.meta}
\usepackage{sagetex}

%The author can find the documentation of additional supporting files from "http://www.ctan.org"

% *** Do not adjust lengths that control margins, column widths, etc. ***

\newtheorem{theorem}{Theorem}
\newtheorem{condition}{Condition}

\begin{document}
%\begin{CJK}{UTF8}{ipxm}

\title{Joint Research of Optics and Fluid Interface}

%%%% To generate auto affiliation numbers please use \author{}\affil{} command

\author{Doumu/Fudepia}
\affil{\email{fudepia@outlook.jp}}

\begin{abstract}
    The abstract text goes here. The abstract text goes here. The abstract text goes here.
    The abstract text goes here. The abstract text goes here. The abstract text goes here.
    The abstract text goes here. The abstract text goes here. The abstract text goes here.
\end{abstract}


\subjectindex{xxxx, xxx}

\maketitle

\section{Differential Analytic}

\begin{figure}
    \usetikzlibrary{intersections}
\begin{tikzpicture}
\tikzstyle{every node}=[font=\fontsize{12}{12}\selectfont] %https://stackoverflow.com/a/56999818/8460574

\coordinate (TL) at (-45,12);
\coordinate (TR) at (45,12);
\coordinate (BL) at (-20,-12);
\coordinate (BR) at (20,-12);

%\draw [help lines] (BL) grid (TR);

\draw[opacity=0] (TL)--(TR);
\draw[opacity=0] (BL)--(BR);
%\node (Int) at (0, 0) {};
\coordinate (Int) at (0, 0);
\draw[->, opacity=0.4] (TL)--(Int);
\draw[->, opacity=0.4] (Int)--(BR);
\draw[color=red] (-45,0)--(45, 0)--node[above] {Ideal Interface}++(-60,0);

\path[draw, name path = realPath] (TL)..controls(Int)..(BR);

\coordinate (LL) at (-45,2);
\coordinate (LR) at (45,2);
\draw[color=violet, name path=Arb] (LL)--(LR)--node[above] {Arbritary}++(-45,0);
\path [name intersections = {of = realPath and Arb, by = IntArb}];
\node at (IntArb) {};
\draw (IntArb)--(IntArb)++(0, 20);

\end{tikzpicture}
\end{figure}
\begin{equation}
    \forall i\in\mathbb{N}:\,
    \frac{\sin\theta_{2i-1}}{\sin\theta_{2i}}=\frac{v_{2i-1}}{v_{2i}}
\end{equation}

Let the ray inbound from upper toward lower and refracs at origin. Let $\theta(y)$ be the acute angle between the velocity vector at that instant and the y-axis.

Let $y=P(x)$ be the path light actually taken, now our goal is to solve $P(x)$ and also the function of speed of light within specific y $v(y)=v(P(x))$.

\begin{equation}
    \lim_{\Delta x\to 0}
    \frac{\sin\theta(P(x))}{\sin(P(x+\Delta x))}=\frac{v(P(x))}{v(P(x+\Delta x))}
\end{equation}

\begin{equation}
    \lim_{\Delta y\to 0}
    \frac{\sin\theta(y)}{\sin(y+\Delta y)}=\frac{v(y)}{v(y+\Delta y)}
\end{equation}

\begin{equation}
    \lim_{r\to 0}
    \frac{\sin\theta(y)}{\sin\left(\theta(y)+r\theta'(y)\right)}=
    \frac{v(y)}{v(y)+rv'(y)}
\end{equation}
\begin{equation}
    \lim_{r\to 0}
    {\sin\theta(y)}{v(y)+rv'(y)}=
    {v(y)}{\sin\left(\theta(y)+r\theta'(y)\right)}
\end{equation}

\newpage

\section{Setup}

Let $\Delta$ be the total width the light travels with path not being straight, hence:

\begin{equation}
\forall x\notin[-\frac{\Delta}{2}, \frac{\Delta}{2}]: \frac{d^2}{dx^2}y=0
\end{equation}

Now define the light ray incoming from left-hand side towards right-hand side and (suppose transition layers' width approaches zero) refracs at origin. And let $h$ be the thickness of the transition layers.

Now let's assume the incoming transition layer has the same thickness as outbounding transition layer, hence we get:

\begin{equation}
\Delta=\frac{h}{2}(\tan\theta_{in}+\tan\theta_{out})
\end{equation}

Now we can first confirm our path function is a function of $\theta_{in}, \theta_{out}, h$

\begin{equation}
P(\theta_{in}, \theta_{out}, h)=
\end{equation}

\section{Continuous Geometric Mean}
\begin{equation}
v(x)=
\begin{cases}
a, & x<-\frac{h}{2} \\
b, & x>\frac{h}{2} \\
a\sqrt[h]{(\frac{b}{a})^{\left(x+\frac{h}{2}\right)}} & \text{otherwise}
\end{cases}
\end{equation}

\section{Analyzing}

Given a full path from $v_1$ to $v_n$, whose path will be $y(v_1, v_n, h)(x)$, we could seperate it into different sections.

\begin{equation}
\underset{i\to i+1}{\Delta}=\frac{h}{n-1}\frac{1}{2}[-\tan\theta_i, \tan\theta_{i+1}]
\end{equation}
\[
\sum_{i=1}^{n-1}\underset{i\to i+1}\Delta=\frac{h}{2}(\tan\theta_1+\tan\theta_n)
\]

\begin{equation}
\forall x\in\underset{i\to i+1}{\Delta}: 
\end{equation}
$(x, y(v_1, v_2, h_{12})(x))+(\underset{2\to{n}}{\Delta}, \underset{2\to{n}}{h})$


\newpage
\section{Finding path}
\subsection{Constructing Layers}
Let function $v(\theta_{in}, \theta_{out}, \Delta, \delta)$ be the velocity of layer $\delta$ (from the entry interface), whose entry velocity is $\theta_{in}$, exit velocity is be $\theta_{out}$, and the whole layer thickness is $\Delta$.\footnote{Here we suppose every layer paraelles}

%\begin{equation}
%\frac{v(\theta_{in}, \theta_{out}, \Delta, 0)}{v(\theta_{in}, \theta_{out}, \Delta, \Delta)}=\frac{\sin(\theta_{in})}{\sin(\theta_{out})}
%\end{equation}

\begin{subequations}
\begin{equation}
a=v(x, y, \beta-\alpha, \alpha)
\end{equation}
\begin{equation}
b=v(x, y, \beta-\alpha, \beta)
\end{equation}
\begin{equation}
\alpha\leq\beta
\end{equation}
\end{subequations}

Let $a=v(x, y, \alpha)$ and $b=v(x, y, \beta)$ ($\alpha\leq\beta$), $\forall p\in[\alpha, \beta]: v(a, b, p-\alpha)=v(x, y, p)$

\subsection{Fractal Structure}
\begin{equation}
    \text{Let }P(\theta_{in}, \theta_{out}, h)\text{ be the path taken}
\end{equation}
\begin{equation}
    P(\theta_{in\to\alpha}, \alpha)=P(\theta_{in}, \theta_{\alpha}
\end{equation}


\newpage
\section{Regarding Special Relativity}
So first let the real velocity \(v_リ\), whom maintains the linear properties of traditional non-relativistic velocity.


Linear to non-linear (relativistic) velocity:
\begin{equation}
    v_S=\frac{v_\jp{リ}\jp{エ}_S\jp{リ}}{\jp{エ}_SS}=v_\jp{リ}\jp{エ}_S\jp{リ}
\end{equation}

\subsection{Constant force/acceleration}
Let a point located at origin, with initial velocity $v(0)=0$.
\begin{subequations}
    \begin{equation}
        a_\jp{ヒ}(v, F, m_{rest})=\frac{F}{m_{rest}}\sqrt{1-\frac{v^2}{c^2}}^3
    \end{equation}
    \begin{equation}
        a_\jp{リ}(t)=k
    \end{equation}
    \begin{equation}
        v_\jp{ヒ}(t)=v_\jp{リ}\jp{エ}_S\jp{リ}=kt\jp{エ}_S\jp{リ}
    \end{equation}
    \begin{equation}
        v_\jp{ヒ}(t)=\int \frac{F}{m_{rest}}\sqrt{1-\frac{v_\jp{ヒ}(t)^2}{c^2}}^3
    \end{equation}
\end{subequations}


\section{Model of Particles}

\begin{tikzpicture}[scale=.8,every node/.style={minimum size=1cm},on grid]
    %
    \begin{scope}[
            yshift=-83,every node/.append style={
                yslant=0.5,xslant=-1},yslant=0.5,xslant=-1
        ]
        \draw[step=4mm, black] (0,0) grid (5,5);
        \draw[black,thick] (0,0) rectangle (5,5);%borders
    \end{scope}
    %
    \begin{scope}[
            yshift=0,every node/.append style={
                yslant=0.5,xslant=-1},yslant=0.5,xslant=-1
        ]
        \fill[white,fill opacity=0.9] (0,0) rectangle (5,5);
        \draw[step=4mm, black] (0,0) grid (5,5); %grid definition
        \draw[black,thick] (0,0) rectangle (5,5);%borders
    \end{scope}
    %
    \begin{scope}[
            yshift=83,every node/.append style={
                yslant=0.5,xslant=-1},yslant=0.5,xslant=-1
        ]
        \fill[white,fill opacity=0.9] (0,0) rectangle (5,5);
        \draw[step=4mm, black] (0,0) grid (5,5);
        \draw[black,thick] (0,0) rectangle (5,5);%borders
    \end{scope}
    %
    \begin{scope}[
            yshift=166,every node/.append style={
                yslant=0.5,xslant=-1},yslant=0.5,xslant=-1
        ]
        \fill[white,fill opacity=0.9] (0,0) rectangle (5,5);
        \draw[step=4mm, black] (0,0) grid (5,5);
        \draw[black,thick] (0,0) rectangle (5,5);%borders
    \end{scope}
    %
    \draw[thick,gray!70!black](5,7) node (Mu) {Mu};
    \draw[thick,gray!70!black](5,4) node (Ko) {Ko};
    \draw[thick,gray!70!black](5,1) node (Fu) {Fu};
    \draw[thick,cyan](-5,4) node[left] (Type) {d};
    \draw
    (Type) edge[cyan,thick,-latex,out=70,in=-245] (0,4.5)
    (Type) edge[cyan,thick,-latex,out=-12,in=154] (0,1.7)
    (Type) edge[cyan,thick,-latex,out=-70,in=170] (0,-1.4);
    \draw[thick,gray!70!black](5,-2) node (Mi) {Mi};
    \draw
    (Fu) edge[bend left,-{Latex[length=2mm]}] node [right] {Project} (Mi)
    (Ko) edge[bend right,-{Latex[length=2mm]}] node [right] {Converge} (Mu)
    (Fu) edge[bend right,-{Latex[length=2mm]}] node [right] {?} (Ko);
    %
\end{tikzpicture}





\section{Conclusion}
The conclusion text goes here.

\section*{Acknowledgment}

Insert the Acknowledgment text here.

% can use a bibliography generated by BibTeX as a .bbl file
% BibTeX documentation can be easily obtained at:
% http://www.ctan.org/tex-archive/biblio/bibtex/contrib/doc/

%\bibliographystyle{ptephy}
%\bibliography{sample}
%
% once the .bbl file has been generated then place the text in your article.

%This is added by T. Yoneya (editor-in-chief) on 2020/07/09.

\let\doi\relax

%without this code before the command "\begin{thebibliography}{}" , an error will be %flagged. When the bibliography is provided as separate .bib file, then this code %should be placed above the commands "\bibliographystyle{}" and "\bibliography{}" %inside the main TeX file. 

\begin{thebibliography}{9}

    \bibitem{mucoe}
        \url{https://www.mail-archive.com/dou-geometry@googlegroups.com/msg00004/____________-___________________________.docx}

\end{thebibliography}

\appendix

\section{Playing with Bezier Curve (failed)}
First setup the environment
\begin{sageblock}
h, t=var("h t")
hh=h/2
theta=var("theta")
tprime=var("tprime", latex_name=r"\theta^\prime")
assume(h, "constant")
assume(h>0)
assume(0<theta, theta<pi/2)
assume(0<tprime, tprime<pi/2)
\end{sageblock}

And now we could try to solve functions like $y(x)$ from parametric bezier curve:

\begin{sageblock}
# x(t)=(t-1)^2*tan(theta)*hh*-1+t^2*tan(tprime)*hh
y=(t-1)^2*hh-t^2*hh
\end{sageblock}

First we try to find $t(x)$
\begin{sageblock}
result=[]    
[((abs(res.rhs()(x=tan(theta)*hh*-1)(h=2, theta=0.2, tprime=0.4))<1e-12 and
 abs(res.rhs()(x=tan(tprime)*hh)(h=2, theta=0.2, tprime=0.4)-1)<1e-12) and
 result.append(res))
 for res in solve((t-1)^2*tan(theta)*hh*-1+t^2*tan(tprime)*hh==x, t)]
t=result[0].rhs() if len(result)==1 else None
\end{sageblock}
Where we get two solutions and only one will satisfied our assumptions (the first print is $0$ and the second yields $1$\footnote{Here we use $EPS=1e-12$ as we're testing it with arbritarily chosen values which makes them no longer symbolic.}), which is:

\[t(x)=\sage{t}\]

And substituting back to $y(t)$ we could get

%\[\sage{y(t=t)}\]


\section{Appendix head}



%\end{CJK}
\end{document}
